\subsection{Example}

We conclude by employing the constructed operators and the tensor algebra of the virtual memory space to implement a minimal differentiable programming space \emph{dCpp}.

\begin{lstlisting}
namespace dCpp{
    var sin(const var& v);
    tau cos;
    tau e;
    tau ln;
    var cos_primitive(const var& v);
    var ln_primitive(const var& v);
    var e_primitive(const var& v);
}
\end{lstlisting}

\noindent We construct the map $sin$ explicitly for educational purposes.

\begin{lstlisting}
var dC::sin(const var& v){
    var out;
    out.id = std::sin(v.id);
    out.order = v.order;
    if(v.order>0){
    	for_each_copy(..., mul_make_pair<std::pair<var*, var> >,
    		dCpp::cos(v.reduce()));
    }
    return out;
}
\end{lstlisting}
Other maps are constructed by employing the operator \eqref{eq:rekurzija}.
\begin{lstlisting}
typedef var (*dTau)(var);
typedef double (*F)(double);

var dCpp::cos_primitive(cont &var v){
    return (-1)*sin(v);
}

dCpp::cos = tau<dTau, F>(std::cos, dCpp::cos_primitive);

var dCpp::ln_primitive(cont &var v){
    return 1/v;
}

dCpp::ln = tau<dTau, F>(std::ln, dCpp::ln_primitive);

dCpp::e_primitive(cont &var v){
    return e(v);
}

tau dCpp::e = tau<dTau, F>(std::e, dCpp::e_primitive);

\end{lstlisting}
