\section{Virtual memory space}\label{sec:virtualMemory}

We model an element of the virtual memory space $v\in\VV_n$ \cite[Definition~4.1]{OperationalCalculus}
\begin{eqnarray}
\VV_{n}=\VV_{n-1}\oplus(V_{n-1}\otimes\VV^*) \label{eq:V_n}
\end{eqnarray}
\begin{equation}
\VV_{n}=\VV\oplus\VV\otimes\VV^*\oplus\cdots\oplus\VV\otimes\VV^{*n\otimes} \implies\VV_n=\VV\otimes T(\VV^*)
\end{equation}
with the class $var$.

\begin{lstlisting}
template<class V>
class var
{
    public:
    	int order;
        V id;
        std::shared_ptr<std::map<var*,var> >* dTau;

        var();
        var(V id);
        var(const var& other);
        ~var();
        void init(int order);
        var d(var* dVar);
        //declerations of algebraic operations
        //declerations of order logic
};
\end{lstlisting}

The virtual memory space $\VV_n$ is the tensor product of the virtual memory $\VV$ with the tensor algebra of its dual. Tensor products of the virtual memory $\VV$ with its dual are modeled using maps, denoted by \emph{dTau}. The address \emph{var*} stands for the component of $v\in\VV_{n-1}$ with which the tensor product of the component of $v^*\in\VV^*$ was computed to generate $v\in\VV_n$ in equation \eqref{eq:V_n}. This depth is contained in the \emph{int order}.

 \begin{izrek}
 An instance of the class $var$ is an element of the virtual memory $\VV_n$.
  \begin{equation}
  var\in\VV_n
  \end{equation}
 \end{izrek}
 \begin{proof}
 \begin{equation}
  id\in \VV\land  dTau\in \VV_{n-1}
  \end{equation}
  $$\land$$
  \begin{equation}
  var=id\oplus dTau\implies var\in\VV_n
  \end{equation}
 \end{proof}

\subsection{Initialization}

A constant element $v_0$ of the virtual memory is an element of $\VV_0=\VV<\VV_n$. We initialize an element to be $n$-differentiable, by mapping
\begin{equation}
init:\VV\times\mathbb{N}\to\VV_n,
\end{equation}
\begin{equation}
v_0.init(n)=v_n\in\VV_n
\end{equation}
The image $v_n$ is an element of $\VV_1=\VV\otimes\VV^*$ with a natural inclusion in $\VV_n$.
\begin{equation}
v_n=(v_0\in\VV)+(\delta^i_j\in\VV\otimes\VV^ {*\otimes})+\sum\limits_{i=2}^n(0\in\VV\otimes\VV^ {*i\otimes}),
\end{equation}
where $\delta^i_j$ is the identity.

\subsection{Algebra over a field}\label{sec:Algebra}

 An algebra over a field is a vector space equipped with a bilinear product. Thus, an algebra is an algebraic structure which consists of a set, together with operations of multiplication, addition, and scalar multiplication by elements of the underlying field. \cite[p.~3]{Algebra}
 
We will construct the algebra by mimicking the application of a direct sum of operators $\sumd_n$ \eqref{eq:sumd} mapping $\dP_0\oplus\dP_0\to\dP_n$ to the maps of scalar multiplication, addition and a bilinear product. 
Their differentiable compositions are modeled by mimicking projections of the operator of program composition \cite[Theorem~5.3]{OperationalCalculus}
 \begin{equation}\label{eq:kom}
   \exp(\D_fe^{h\D_g}): \dP\to \dP\to\dP_\infty.
   \end{equation}
and fixing one of the mappings \cite[eq.~44,~45]{OperationalCalculus}. If one would prefer the implementation through application of operators to appropriate mappings, to the explicit implementation of this section, one can do so with the operators from section \ref{sec:operators}.

\subsubsection{Vector space over a field $K$}\label{sec:vectorSpace}
 
As part of the construction of an algebra, we construct a vector space $\VV_n$ over a field $K$. A vector space is a collection of objects that can be added together and multiplied with elements of the underlying field $K$.

We begin with scalar multiplication.

\begin{lstlisting}
template<class K>
var var::operator*(K n)const{
    var out;
    out.id=this->id*n;
    for_each_copy(...,mul_make_pair<pair<var*,var> >, n);
    return out;
}

template<class K>
var var::operator/(K n)const {...};
\end{lstlisting}
Scalar multiplication and its convenient inverse employ the function \emph{for\_each\_copy}, applying the provided operation 
\begin{lstlisting}
template<class V, class K>
V mul_make_pair(V v, K n) {
    return std::make_pair(v.first, v.second * n);
}
\end{lstlisting}
to each one of the components of $this$ and storing the result in $out.dTau$.

Vector addition by component is implemented by 
\begin{lstlisting}
var var::operator+(const var& v)const{
    var out;
    out.id=this->id+v.id;
    merge_apply(..., sum_pairs<pair<var*, var> >);
    return out;
}
\end{lstlisting}
Vector addition by component employs the function \emph{merge\_apply}, applying the provided function \emph{sum\_pairs}
\begin{lstlisting}
template<class V>
T sum_pairs(V v1, V v2) {
    return std::make_pair(v1.first, v1.second + v2.second);
}
\end{lstlisting}
to corresponding components, storing the result in $out.dTau$.

\begin{izrek}
Class $var$ models a vector space over a field $K$.
\end{izrek}
\begin{proof}
By implementations of addition by components and multiplication with a scalar $k\in K$, the axioms of the vector space are satisfied.
\end{proof}

\subsubsection{Algebra over a field $K$}\label{sec:algebra}

To construct an algebra over a field $K$, we equip the vector space $\VV_n$ with a bilinear product by components.

\begin{lstlisting}
var var::operator*(const var& v)const{
    var out;
    out.id=this->real*v.id;
    out.order=this->order<v.order?this->order:v.order;
    if(out.order>0){
        map<int,double> tmp1;
        map<int,double> tmp2;
        for_each_copy(...,mul_make_pair<pair<var*,var> >, 
          v.reduce());
        for_each_copy(...,mul_make_pair<pair<var*,var> >, 
          this->reduce());
        merge_apply(..., sum_pairs<pair<var*,var> >);
    }
    return out;
}
\end{lstlisting}
The employed functions have been explained at previously use. The function \emph{reduce} makes a shallow copy of $this$, while reducing the order of the returned copy (making it one time less differentiable). This bilinear product contains Leibniz rule within its structure, as the projection of the operator $\exp(\D_fe^{h\D_g})(g): \dP\to\dP_\infty(g)$ to the unit $n$-cube was applied to the algebra.

\begin{izrek}
Class $var$ models an algebra over a field $K$.
\end{izrek}
\begin{proof}
By the implementation of a bilinear product by components the axioms of an algebra over a field are satisfied.
\end{proof}

We may now trivially implement the remaining operators existing in the algebra.

\begin{lstlisting}
var var::operator-(const var& v)const{
    return *this+(-1)*v;
}

var var::operator/(const var& v)const{
    return *this*(v^(-1));
}
\end{lstlisting}
Similarly, the following operators can be assumed to be generated by the existing algebra.

\begin{lstlisting}
var operator*(double n)const;
var operator+(double n)const;
var operator-(double n)const;
var operator/(double n)const;
var operator^(double n) const;
var& operator=(const var& v);
var& operator+=(const var& v);
var& operator-=(const var& v);
var& operator*=(const var& v);
var& operator/=(const var& v);
var& operator*=(double n);
var& operator/=(double n);
var& operator+=(double n);
var& operator-=(double n);
var operator*(double n, const var& v);
var operator+(double n, const var& v);
var operator-(double n, const var& v);
var operator/(double n, const var& v);
var operator^(double n, const var& v);
\end{lstlisting} 

Order logic is trivially implemented, by mapping
\begin{equation}
\VV.id\times\VV.id\to\{0,1\}.
\end{equation}
\begin{lstlisting}
bool operator==(const var& v)const;
bool operator!=(const var& v)const;
bool operator<(const var& v)const;
bool operator<=(const var& v)const;
bool operator>(const var& v)const;
bool operator>=(const var& v)const;
\end{lstlisting}

