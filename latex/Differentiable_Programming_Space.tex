\section{Differentiable programming space}\label{sec:differentiableProgSpace}

With the algebra over $\VV_n$ implemented, we turn towards constructing a differentiable programming space, from the programming space
              
\begin{equation}
\CC:\VV\to\VV.
\end{equation}

\begin{definicija}[Differentiable programming space]\label{def:dP}
  A \emph{differentiable programming space} $\dP_0$ is any subspace of $\F_0:\VV\to\VV$ such that
  \begin{equation}\label{eq:P}
  \D\dP_0\subset\dP_0\otimes T(V^*)
  \end{equation}
  The space $\dP_n<\F_n:\VV\to\VV_n$, spanned by $\{\D^k\dP_0;\quad 0\le k\le n\}$ over $K$, is called a differentiable programming space of order $n$. When all elements of $\dP_0$ are analytic, we denote $\dP_0$ as an \emph{analytic programming space}. \cite[Definition~4.2]{OperationalCalculus}
 \end{definicija}
              
\begin{izrek}[Infinite differentiability]\label{izr:P}
  Any differentiable programming space $\dP_0$ is an
  infinitely differentiable programming space, meaning that
  \begin{equation}\label{eq:P_n}
      \D^k\dP_0\subset\dP_0\otimes T(V^*)
    \end{equation}
for any $k\in\mathbb{N}$. \cite[Theorem~4.1]{OperationalCalculus}
\end{izrek}

It follows from Theorem \ref{izr:P} that a differentiable programming space of order $n$, $\dP_n:\VV\to\VV\otimes T(\VV^*)$ can be embedded into the tensor product space of the programming space $\dP_0$ and the space $T_n(\VV^*)$ of multi-tensors of order less than equal $n$ \cite[Corollary~4.1.1]{OperationalCalculus}. This means that all that the tuple $(\VV,\dP_0)$ - together with the structure of the tensor algebra $T(\VV^*)$ - is sufficient for constructing arbitrary differentiable programming spaces $\dP_n$.
              

             
\subsection{Virtual tensor machine}\label{sec:analyticVmachine}

In this subsection we provide a minimal implementation of a virtual tensor machine, together with the pullback of \eqref{eq:sumd} through an arbitrary program. This is equivalent to the projection of the operator of program composition to the unit hyper-cube. 

\begin{definicija}[Virtual tensor machine]
The tuple $M=\langle \VV,\dP_0\rangle$ is an analytic, infinitely  differentiable virtual machine, where
   
    \begin{itemize}
    \item
    $\VV$ is a finite dimensional vector space
    \item
    $\VV\otimes T(\VV^*)$ is the virtual memory space
    \item
    $\dP_0$ is an analytic programming space over $\VV$.
    \end{itemize}
  \end{definicija}

  \subsection{Operators}\label{sec:operators}

The operator of tensor series expansion \cite[Theorem~5.1]{OperationalCalculus}

\begin{equation}\label{eq:specProg}
            e^{h\D}:\dP\to\VV\to \VV\otimes \T(\VV^*),
          \end{equation}

\begin{equation}
  e^{h\D}=\sum\limits_{n=0}^{\infty}\frac{(h\D)^n}{n!}
 \end{equation}
 is evaluated at $h=1$ and projected onto the unit $N$-cube, arriving at 
\begin{equation}\label{eq:DD}
    \sumd_N = 1+\D +\D^2 +\ldots + \D^N.
  \end{equation}
The expression \eqref{eq:DD} obeys the recursive relation
\begin{equation}
      \label{eq:rekurzija}
      \sumd_{k+1} = 1+\D\sumd_k,
    \end{equation}
which can be used for constructing programming spaces of arbitrary order.


which is used to implement $\DD^n\CC\subset\dP_n$, employing the recursive relation for increasing the order of $\CC$ \cite[Proposition~5.1]{OperationalCalculus}.
\begin{equation}
      \label{eq:rekurzija}
      \sumd_{k+1} = 1+\D\sumd_k,
    \end{equation}
\begin{equation}
\sumd_{k+1}\CC = \CC+\D\sumd_k\CC.
\end{equation}
        
        The \emph{double id} stands for the identity operator $1$ in the expression and the \emph{map<var*, var> dTau} for the operator $\D\sumd_k$.
        
        Compositions of these operators is achieved by mimicking the generalized pullback operator \cite[Remark~11]{OperationalCalculus}
        \begin{equation}\label{eq:opKompo}
          \exp(\D_fe^{h\D_g})(g): \dP\to\dP_\infty(g),
          \end{equation}
          projected onto the unit N-cube.
          
          \begin{opomba}
          Implementation in this paper is intended to be simply understood and is written as such. But, with the existing algebra and operational calculus, one could easily implement the operator $\exp(\D_fe^{h\D_g})(g)$ by any of the efficient techniques available, as it is given by a generating function.
          \end{opomba}
        
\begin{lstlisting}
template<class dTau, class K>
class tau
{
public:
    tau();
    tau(K mapping, dTau primitive);
    ~tau();
    var operator()(const var&v);
private:
    dTau primitive;
    K mapping;
};
\end{lstlisting}        
        
\begin{lstlisting}
var tau::operator()(const var&v){
    var out;
    out.id=mapping(v.id);
    for_each_copy(...,mul_make_pair<std::pair<var*,var> >, 
      primitive(v.reduce()));
    return out;
}
\end{lstlisting}
