\section{Differentiable programming space}\label{sec:differentiableProgSpace}

With the algebra over $\VV_n$ implemented, we turn towards constructing a differentiable programming space, from the programming space
              
\begin{equation}\label{eq:cc}
\CC:\VV\to\VV.
\end{equation}

\begin{definicija}[Differentiable programming space]\label{def:dP}
  A \emph{differentiable programming space} $\dP_0$ is any subspace of $\F_0:\VV\to\VV$ such that
  \begin{equation}\label{eq:P}
  \D\dP_0\subset\dP_0\otimes T(V^*)
  \end{equation}
  The space $\dP_n<\F_n:\VV\to\VV_n$, spanned by $\{\D^k\dP_0;\quad 0\le k\le n\}$ over $K$, is called a differentiable programming space of order $n$. When all elements of $\dP_0$ are analytic, we denote $\dP_0$ as an \emph{analytic programming space}. \cite[Definition~4.2]{OperationalCalculus}
 \end{definicija}
              
\begin{izrek}[Infinite differentiability]\label{izr:P}
  Any differentiable programming space $\dP_0$ is an
  infinitely differentiable programming space, meaning that
  \begin{equation}\label{eq:P_n}
      \D^k\dP_0\subset\dP_0\otimes T(V^*)
    \end{equation}
for any $k\in\mathbb{N}$. \cite[Theorem~4.1]{OperationalCalculus}
\end{izrek}

It follows from Theorem \ref{izr:P} that a differentiable programming space of order $n$, $\dP_n:\VV\to\VV\otimes T(\VV^*)$ can be embedded into the tensor product space of the programming space $\dP_0$ and the space $T_n(\VV^*)$ of multi-tensors of order less than equal $n$ \cite[Corollary~4.1.1]{OperationalCalculus}. For the specific case of \eqref{eq:cc} this entails
\begin{equation}
\DD^n\CC<\dP_n\iff\DD^n\CC\subset\CC\otimes T(\VV^*)
\end{equation}
which we ensure by providing closure of \eqref{eq:cc} under $\D$. Furthermore, this means that the tuple $(\VV,\dP_0)$
\footnote{In our case $\dP_0:=\CC$ \eqref{eq:cc}.}
 - together with the structure of the tensor algebra $T(\VV^*)$ - is sufficient for constructing arbitrary differentiable programming spaces $\dP_n$ .
              
             
\subsection{Operators}\label{sec:operators}

In this subsection we provide a simple implementation of \eqref{eq:sumd} and its pullback through an arbitrary program. This is equivalent to the projection of the operator of program composition to the unit hyper-cube. As we have already constructed the algebra of the memory $\VV_n$, and provided a differentiable programming space $\CC$, such an operator will enable us to arbitrarily increase its order.

The operator of tensor series expansion \cite[Theorem~5.1]{OperationalCalculus}

\begin{equation}\label{eq:specProg}
            e^{h\D}:\dP\to\VV\to \VV\otimes \T(\VV^*),
          \end{equation}

\begin{equation}
  e^{h\D}=\sum\limits_{n=0}^{\infty}\frac{(h\D)^n}{n!}
 \end{equation}
 is evaluated at $h=1$ and projected onto the unit $N$-cube, arriving at 
\begin{equation}\label{eq:DD}
    \sumd_N = 1+\D +\D^2 +\ldots + \D^N.
  \end{equation}
The expression \eqref{eq:DD} obeys the recursive relation
\begin{equation}
      \label{eq:rekurzija}
      \sumd_{k+1} = 1+\D\sumd_k,
    \end{equation}
which can be used for constructing programming spaces of arbitrary order. Note that only explicit knowledge of $\tau_1:\dP_0\to\dP_1$ is required \cite[Proposition~5.1]{OperationalCalculus}. We perform its pullback through an arbitrary program by mimicking the operator of program composition \cite[Theorem~5.2]{OperationalCalculus}
\begin{equation}\label{eq:opKompo}
  \exp(\D_fe^{h\D_g}): \dP\to\dP\to\dP_\infty,
  \end{equation}
partially evaluating on the second map and projecting it onto the unit $N$-cube.         
        
\begin{lstlisting}
template<class dTau, class K>
class tau
{
public:
    tau();
    tau(K mapping, dTau primitive);
    ~tau();
    var operator()(const var&v);
private:
    dTau primitive;
    K mapping;
};
\end{lstlisting}        
        
\begin{lstlisting}
var tau::operator()(const var&v){
    var out;
    out.id=mapping(v.id);
    for_each_copy(...,mul_make_pair<std::pair<var*,var> >, 
      primitive(v.reduce()));
    return out;
}
\end{lstlisting}
