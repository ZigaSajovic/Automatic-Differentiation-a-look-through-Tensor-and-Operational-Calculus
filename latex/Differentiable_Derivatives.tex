\subection{Differentiable Derivatives}\label{sec:orderReduction}
 
The ability to treat the $k$-th derivative of a program, as part of a different differentiable program appears useful in many fields. To that purpose, we must be able to threat the derivative itself as a differentiable program. This is easily implemented given our simplified construction, as we are working solely with the projections of the operators to the unit $N$-cube, which eases our troubles.

\begin{izrek}\label{izr:reductionMap}
There exists a reduction of order map $\phi:\dP_n\to \dP_{n-1}$, such that the
following  diagram commutes
\begin{equation}\label{eq:reductionMap}
\begin{tikzcd}
  \dP_n \arrow{r}{\phi} \arrow{d}{\D} & 
  \dP_{n-1} \arrow{d}{\D}\\
  \dP_{n+1} \arrow{r}{\phi} & 
  \dP_{n}
\end{tikzcd}
\end{equation}
satisfying
\begin{equation}
\forall_{P_1\in\dP_0}\exists_{P_2\in\dP_0}\Big(\phi^k\circ \sumd_n(P_1)=\sumd_{n-k}(P_2)\Big)
\end{equation}
for each $n\ge 1$.
\end{izrek}  
\begin{corollary}\label{cor:extraxtDerivatives}
By Theorem \ref{izr:reductionMap}, $n$-differentiable $k$-th derivatives of a program $P\in\dP_0$ can be extracted by
\begin{equation}
^{n}P^{k\prime}=\phi^k\circ \sumd_{n+k}(P)\in\dP_n
\end{equation}
\end{corollary}

Given that we are only interested in computing (differentiable) derivatives of programs (i.e. automatic differentiation), and do not need the algebra of higher-order constructs the full treatment of operational calculus would afford us, we are working entirely on the unite $N$-cube. Hence, our reduction of order map is a simple projection to the $(N-1)$-cube.
